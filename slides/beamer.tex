\documentclass{beamer}

% Thème sobre et académique
\usetheme{Madrid}
\usecolortheme{seahorse}

% Langue et encodage
\usepackage[utf8]{inputenc}
\usepackage[french]{babel}
\usepackage{lmodern}

% Couleur personnalisée
\definecolor{c}{RGB}{120,20,30}
\setbeamercolor{title}{fg=white,bg=c}
\setbeamercolor{frametitle}{fg=white,bg=c}
\setbeamercolor{structure}{fg=c}

% Informations de la présentation
\title[Apprentissage de classes déséquilibrées]{\textbf{Apprentissage de classes déséquilibrées}}
\subtitle{\LARGE HAX907X - Apprentissage statistique}
\author[]{\textbf{SAWADOGO Kader \\ GERMAIN Marine \\ LABOURAIL Célia \\ MARIAC Damien}}
\institute[Université Montpellier]{Université Montpellier \\ Département de Mathématique}
\date{\today}

\AtBeginSection[]
{
  \begin{frame}{Sommaire}
    \tableofcontents[currentsection]
  \end{frame}
}

\begin{document}

\begin{frame}
  \titlepage
\end{frame}
\section{Contexte}

\section{Problématique}

\begin{frame}{Les classes déséquilibrées}

    \textbf{Problème :} difficulté à prédire la classe minoritaire \\[0.3cm]
    $\Rightarrow$ Le modèle a tendance à ignorer cette classe. \\[0.5cm]

    \begin{block}{Exemple général}
        \begin{itemize}
            \item 99\% vs 1\%.
            \item Un modèle naïf prédit la classe majoritaire à une précision de 99\,\%.
            \item Mauvais modèle.
        \end{itemize}
    \end{block}

\end{frame}


\begin{frame}{Problématique scientifique}
    \centering
    \Large{Comment atténuer le déséquilibre des classes pour améliorer la performance réelle du modèle ?}
\end{frame}


\section{Méthodes}


\section{Limites des méthodes}


\section{Application}


\section{Conclusion}

\begin{frame}{Bilan général des méthodes}
    \centering
    \small
    \begin{tabular}{|l|p{3cm}|p{3cm}|}
        \hline
        \centering
        \textbf{Méthode} & \textbf{Points forts} & \textbf{Limites} \\
        \hline
        \centering
        ROS & Simplicité, conserve toutes les données & Overfitting, grand volume de données \\
        \hline
        \centering
        RUS & Rapide et réduit le biais & perte d'information et représentativité \\
        \hline
        \centering
        SMOTE & Données synthétiques variées & Coût élevé, sensible aux outliers \\
        \hline
    \end{tabular}
    \vspace{0.7cm}

    \textbf{Aucune méthode n’est universelle :}\\
    le choix dépend du jeu de données et du modèle.
\end{frame}

\begin{frame}{Conclusion et perspectives}
    \begin{itemize}
        \item Pour notre jeu de données, la méthode la plus efficace est SMOTE.\\[0.5cm]
        \item  Pour aller plus loin : il serait pertinent de combiner des méthodes existantes ou de pondérer les modèles.
    \end{itemize}
\end{frame}

\begin{frame}
    \centering
    \Huge Merci pour votre attention !
\end{frame}

\end{document}